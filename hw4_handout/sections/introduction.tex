\section{Introduction}
Graph convolutional networks (GCNs)~\cite{kipf2016semi} have emerged as the architecture of choice for learning meaningful graph representations. 
In this assignment, we will implement a GCN from scratch.
We will then use the node representations learned by a GCN for three downstream tasks: i) node classification, ii) link prediction, and iii) graph classification.

% The goal of this assignment is to help you in understanding a GCN network in detail. 
% We will begin by reviewing the details of a GCN network and follow it up with practical tasks.

\paragraph{Notations}
Let $\mG(\mV, \mE)$ be a graph with node-set $\mV$ and edge-set $\mE$. Let $v \in \mV$ be a vertex in the graph $\mG$.
The adjacency matrix of $\mG$ is denoted by $\mA$.
$|\mV|$ is used to denote the size of the vertex set.


